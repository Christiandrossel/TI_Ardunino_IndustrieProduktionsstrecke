\chapter{Umsetzung}
Hier wird die Umsetzung der einzelnen Bestandteile des Projekts beschrieben. Probleme auf die das Team gestoßen ist und deren Lösung werden ebenfalls erläutert.\todo[author=anja,inline]{das hab ich nicht gemeint (ist aber nicht schlimm), sondern eher sone kurzbeschreibung mit paar fakten wie: modellfabrik in der sich 3 roboter gleichzeitig bewegen. 2 verschiedene arten von stationen, modular, ...}
%Darstellung der konkreten Umsetzung des Projekts inklusive besonderer aufgetretener Probleme/Herausforderungen und deren Lösung. Das Vorgehen sollte so nachvollziehbar sein, dass der Leser das Projekt vom Prinzip genauso umsetzen könnte (meint aber nicht, dass hier jeder Codeschnipsel erscheinen muss). 
%Im letzten Abschnitt könnte als Zusammenfassung die Ergebnisdarstellung erfolgen. Sie könnte aber auch in ein eigenes Kapitel umgewandelt werden, wenn sie umfangreich wird.

\section{Roboter}
Der Roboter ist ein Keystudio Smart Small Turtle Robot, welcher auf einer vorgegebenen Strecke fährt und an bestimmten Stationen halten soll. \todo[author=anja]{Ein Bild von der Strecke wär hier schön}
\subsubsection{Linie folgen}
Die Hauptaufgabe\todo[author=anja]{nur zur kenntnisnahme: Priorität hat vermutlich eher das nicht anstoßen} des Roboters liegt darin, auf einer schwarzen Linie entlang zu fahren, um den Kurs an den Stationen vorbei zu absolvieren. Dafür werden die 3 Infrarotsensoren auf der Unterseite des Roboters genutzt, um fest zu stellen, wo auf der Linie sich der Roboter befindet.

Sollte er an einer Seite die Farbe Weiß fest stellen, so justiert er seine Fahrtrichtung, um im schwarzen Bereich zu bleiben. Wenn alle 3 Sensoren Weiß feststellen, so ist der Roboter über einen Breakpoint gefahren, eine weiße Fläche auf der schwarzen Strecke, an welcher der Roboter mit den Stationen kommuniziert, um seine weiteren Schritte in Erfahrung zu bringen.

\subsubsection{Auffahrunfälle vermeiden}
Während die Roboter auf der vorgesehenen schwarzen Linie fahren, kann es passieren, dass sie auf einen Roboter vor ihnen warten müssen. Um Auffahrunfälle zu vermeiden, sind Ultraschallsensoren an der Vorderseite der Roboter angebracht. Diese registrieren jedoch andere Roboter erst zu spät oder gar nicht, da die Kabel im oberen Bereich des Roboters nicht präzise erkannt werden können.\todo[author=anja]{bissl genauer beschreiben, dass das eigentliche problem in der fehlenden reflexionsfläche liegt}

Das Problem konnte mit einem Schild gelöst werden, welches für jeden Roboter gedruckt und an dessen Rückseite montiert wurde. Dadurch haben die Roboter dahinter ein klar erkennbares Hindernis auf der richtigen Höhe.\todo[author=anja]{Ein Bild von nem Rodoter mit Schild wär hier gut.}

\section{Bohrer}
In dem Projekt werden zwei Bohrer eingesetzt, welche aus den Arduino-Uno-Starterkits zusammen gebaut wurden. Die meisten Bestandteile der Bohrer sind aus dem Starterkit, bis auf das Bluetooth Modul (HC-05), welches nachträglich zur Kommunikation mit den Robotern bestellt wurde.\todo[author=anja,inline]{Bissl konkreter der Aufbau/die Bestandteile der Borher beschreiben. Bild wär auch hier schön}

\subsubsection{Roboter erkennen}
Nachdem der Bohrer vom Roboter die Nachricht bekommen hat, dass dieser zu ihm fährt, bereitet sich der Bohrer auf die Ankunft des Roboters vor und bestätigt keine Anfragen von weiteren Robotern. Durch einen einstellbaren Lichtsensor kann der ankommende Roboter erkannt werden und die Arbeit des Bohrers beginnt.

\subsubsection{Arbeiten}
Die Arbeit des Bohrers wird im Rahmen des Projekts als Fortschrittsbalken auf einem LCD symbolisiert. Der Fortschritt startet sobald der Roboter vor den Bohrer gefahren ist. Nachdem der Fortschrittsbalken voll ist\todo[author=anja]{Wenn dus rauskriegst, könnte man noch erwähnen wie lange das dauert}, ist die Arbeit abgeschlossen und der Bohrer sendet dem Roboter, dass er weiterfahren kann.

\section{Kran}
Der Kran besteht wie der Bohrer aus dem Starterkit und einem zusätzlichen Bluetooth Modul.\todo[author=anja]{anmerkung wie beim bohrer} Ebenfalls erkennt er Roboter die vor fahren wie ein Bohrer mit Hilfe des Lichtsensors.\todo[author=anja]{der satz klingt scheiße :D}
\subsubsection{Arbeiten}
Im Gegensatz zu den Bohrern spielt der Kran bei der Arbeit ein Musikstück ab. Dies wird über ein Piezzo-Element realisiert und startet sobald ein Roboter vor den Kran fährt. Ist das Lied zu Ende, sendet der Kran dem vor ihm stehenden Roboter ein Signal zum Weiterfahren über das Bluetooth Modul.

\section{Kommunikation}
Ein großes Problem bei der Umsetzung war, dass die Bluetooth-Module der Turtle-Roboter (HC-06) nur als Slaves agieren können und damit sich nicht untereinander verbinden können. Dieses Problem wurde durch eine Android-App gelöst.

Diese Android-App\todo[author=anja]{wenn du willst, nochn beispielbild. außerdem beschreiben, was die alles so anzeigt} verbindet sich mit allen Teilnehmern und gibt die gesendeten Daten weiter. Sollte beispielsweise ein Turtle-Roboter an einem Bohrer stehen und dieser seine Arbeit beendet haben, so sendet er eine Nachricht an die App. Die App leitet diese Information an den Roboter weiter, damit dieser seine Fahrt fortsetzen kann.\todo[author=anja]{richtig so?}
\todo[author=anja,inline]{verweis aufs bild noch rein machen (verweis in klammern reicht) und vlt noch kurz was zur art der mitteilung sagen. also sowas wie r's für frei/ready verschickt. das wir zustandsbasiert arbeiten, könnte man noch erwähnen.}

Dazu kommt allerdings ein gewisser Overhead, da der Roboter der App mitteilen muss, an welcher Station er sich gerade befindet, damit die App, weiß an wen sie die Informationen schicken muss.
\begin{figure}[h]
\begin{center}
\includegraphics[width=\textwidth]{img/Ablaufdiagram-App.png}
\caption{Kommunikation der Server-App mit den einzelnen Geräten}
\end{center}
\end{figure}
\todo[author=anja,inline]{kleiner fehler in der abbildung: 1. blaues quadrat: roboter fragt nach freie\emph{m} bohrer}