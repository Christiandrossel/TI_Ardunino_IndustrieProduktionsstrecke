
\condquotebook{\enquote{Hab ich das gesagt, oder hab ich das gesagt?}, hab ich gesagt.}{Ich}{Meine Erinnerungen}
\chapter{Beispielkapitel}


Ein sinnloser\index{sinnlos} Satz \gls{etc.}, der Tabelle~\ref{tab:testtable} und Bild~\ref{fig:testfig} referenziert.
Außerdem zitieren wir~\textcite{test16}.
\todo[author=anja]{Aber jetzt ist wirklich Schluss}Zum Abschluss noch ganz wichtig.\footnote{Wichtigste Stelle in philosophischen Abhandlungen.}


\begin{figure}[p]
    \centering
    \begin{subfigure}[b]{0.48\linewidth}
        \centering
        \includegraphics[scale=0.2]{img/Htw-dresden-logo}
        \caption{Logo der HTW Dresden}\label{fig:logo-htw}
    \end{subfigure}\enspace%
    \begin{subfigure}[b]{0.48\linewidth}
        \centering
        \includegraphics[scale=0.4]{img/elderly-cyborg}
        \caption{Kunstwerk}\label{fig:kunstwerk}
    \end{subfigure}
    \caption{Beispiel zweier Bilder nebeneinander}\label{fig:testfig}
\end{figure}

\begin{figure}[p]
    \centering
    \includegraphics[scale=0.9]{img/fatal}
    \caption{Fataler Fehler}\label{fig:fatal}
\end{figure}

\begin{table}
    \centering
    \caption[Testtable]%
        {Testtabelle: Zeilen und Spalten.}
    \rowcolors{2}{tablecol-odd}{tablecol-even}
    \begin{tabular}{lll}
        \toprule
        Spalte 1 & Spalte 2 & Spalte 3 \\
        \midrule
        Toller Text  & $a$ & $\approx 2$ \\
        Noch tollerer Text & $b$ & $\approx 4$ \\
        Der tollste Text & $c$ & $\approx 5$ \\
        \bottomrule
    \end{tabular}\label{tab:testtable}
\end{table}

\section{Beispielabschnitt}

Hier mal ein Beispiel für ein Codelisting:
\begin{lstlisting}[language=Arduino]
    if (energyState >= 8){
      digitalWrite(3, HIGH);
      digitalWrite(4, HIGH);
      digitalWrite(5, HIGH);
      // let red leds blink fast
      delay(250);
      digitalWrite(4, LOW);
      digitalWrite(5, LOW);
      delay(250);
      energyState--;
    }
\end{lstlisting}
\todo[author=anja, inline, color=green]{Nicht vergessen, den dann da wieder rauszumachen.}
Der gesamte Beispiel-Code befindet sich in Anhang~\ref{ch:AppTest}.

\section{Zweiter Abschnitt: Hinweise}
\subsection{Schreibstil}
Der Schreibstil sollte möglichst sachlich und neutral/objektiv sein. Die Verwendung von umgangssprachlichen Formulierungen ist unpassend. Personalpronomen, insbesondere für die eigene Person, sind zu vermeiden. \enquote{Man} ist keine Alternative. Oft hilft hier eine Passivform. Persönliche Meinungen sind in Argumentationen zu überführen. Die Konsequenzen bei Missachtung der Regeln sind fatal: bedrohte Tierarten sterben aus, der Meeresspiegel steigt, die Warmzeit verlängert sich um jeweils 1000 Jahre, die Sonne dehnt sich aus -- kurz um -- die Erde rückt mit jedem Mal ihrem Untergang näher. Beispiele und Lösungen werden in Tabelle~\ref{tab:bsptable} aufgezeigt. Ausgenommen sind nur Platzhaltertexte und temporäre Kommentare.

\begin{table}
	\centering
	\caption{Beispielformulierungen}
	\label{tab:bsptable}
	\rowcolors{2}{tablecol-odd}{tablecol-even}
	\begin{tabular}{p{7.5cm}p{7.5cm}}
		\toprule
		Beispiel & Bewertung \\
		\midrule
		Wir haben uns gegen einen Zitronenkuchen entschieden, weil der zu säuerlich sein könnte.
		& Oh, nein! $Wir$ haben die Koalabären ausgelöscht. Die waren doch so niedlich. \\
		Da sich Zitrone, aufgrund ihrer Säure, ungünstig auf bereits vorhandene Magenbeschwerden auswirken kann, wurde von Zitronenkuchen abgesehen.
		& Yeah, ein Koala wurde gesichtet. Lebend! \\
		Und dann habe ich das Mehl einfach gesiebt. 
		& Und dann habe ich den Meeresspiegel einfach um 10 cm angehoben. \\
		Anschließend wurde das Mehl gesiebt.
		& Juhu, die Niederlande gibts doch noch. \\
		Statt Selters kann man auch Natron verwenden.
		& Die Kacke ist am Dampfen! \\
		Mineralwasser kann auch durch Natron ersetzt werden.
		&  Doch keine Klokühlung nötig. Wieder was gespart.\\
		\bottomrule
	\end{tabular}
\end{table}

% vim: set spelllang=en_gb
