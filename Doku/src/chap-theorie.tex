\chapter{Verwandte Arbeiten}

Intelligente Produkte, Methoden und Prozesse stehen im Zentrum von Industrie 4.0 \parencite{arbeitsbericht}. Produktionsstrukturen sollen flexibel und anpassbar gestaltet werden. Dazu werden Informations- und Kommunikationstechnologien eingesetzt, welche die digitale und physische Welt miteinander verknüpfen und eine Smart Factory entstehen lassen. Eine flexiblere Reaktion auf geänderte Anforderungen an ein Produkt selbst während der Produktion wird ermöglicht. Dies geschieht insbesondere durch Systeme und Ressourcen die untereinander vernetzt sind und situationsabhängig selbst agieren können.

\textcite{agents} entwickelten ein Konzept zur Selbstorganisation von Agenten und verglichen es mit einer zentralen Produktionsplanung, die einem üblichen Algorithmus folgt. Sie unterscheiden acht Arten von Agenten, die weiterhin nach ihrer Lebensdauer, entweder entsprechend der Aufgabenerfüllung oder entsprechend der Produktion, gruppiert werden können. Jeder Agent erfüllt jeweils eigene Aufgaben und kennt nur seine unmittelbaren Nachbarn. Die Agenten kommunizieren untereinander und arbeiten statusbasiert.

Im Vergleich zeigte sich, dass die selbstorganisierte Planung mit einer höheren Durchlaufzeit und damit verbunden höheren Liegezeiten sowie höheren Lagerbeständen einherging als die zentral geplante Produktion \parencite{agents}. Dagegen erwies sich die selbstorganisierte Planung deutlich robuster und anpassungsfähiger gegenüber Störungen und Abweichungen im Produktionsverlauf.

\textcite{rfid} untersuchten die Möglichkeit des Einsatzes von Radio Frequency Identifikation (RFID) zur Unterstützung von Entscheidungsprozessen und Lokalisation in einer symbolischen Smart Factory. Dazu setzten sie Automated Guided Vehicles (AGVs) ein, die auf einer Teststrecke mit Kreuzungen und Haltepunkten fuhren. AGVs sind fahrerlose Transportfahrzeuge, die sich autonom auf einem vorgefertigten Pfad bewegen.

Als Basis der eingesetzten AGVs diente der Makeblock mbot Bausatz \parencite{rfid}. Er besitzt unter anderem einen auf Arduino Uno basierenden Microcontroller, ein Bluetoothmodul, einen Ultraschallsensor zur Hinderniserkennung und einen Infrarotsensor für Line Tracking. Die AGVs wurden um eine Plattform, mit der kleine Objekte transportiert werden können, und einen RFID-Leser erweitert. Sie bewegen sich auf einer markierten Teststrecke in die an bestimmten Stellen wie Kreuzungen und Haltepunkten RFID-Tags eingebaut wurden. Sie sollen neben einer Lokalisierung der AGVs auch Entscheidungen wie Streckenwahl und Haltepunkte unterstützen.

Die AGVs werden über eine zentrale Einheit in Form einer Android App miteinander verknüpft und gesteuert \parencite{rfid}. Die Kommunikation erfolgt über Bluetooth, wobei die App als Server und die AGVs als Clients agieren. Für die Kommunikation wurde ein einfaches Protokoll entwickelt. Es besteht aus einem Identifikator, einer Instruktion, einem Variablenwert und einem Zeichen zur Kennzeichnung des Nachrichtenendes. Über die App können nicht nur alle AGVs gestartet und gestoppt werden, sondern auch Nachrichten der AGVs dargestellt und gezielt Nachrichten, zum Beispiel um den verfolgten Pfad zu ändern, gesendet werden. Wenn sie keine Instruktion von der zentralen Einheit erhalten, bewegen sie sich selbstständig auf der Teststrecke.

Fährt ein AGV über einen RFID-Tag, liest es diesen aus und erhält einen Schlüssel \parencite{rfid}. Diesen schickt es zum einen an die zentrale Einheit, welche dadurch den Weg des AGVs verfolgen kann. Zum anderen wertet es den Schlüssel aus und passt sein Verhalten in Abhängigkeit des geplanten Wegs an. Das kann entweder das Einschlagen einer bestimmten Richtung an einem Kreuzungspunkt bedeuten oder das Ansteuern eines bestimmten Haltepunkts und Anhaltens für eine festgelegte Zeit.

Wenn ein AGV ein Hindernis erkennt, meldet es dies der zentralen Einheit \parencite{rfid}. Diese wertet dann aus, ob es sich wirklich um ein Hindernis oder um ein anderes AGV handelt. Wurde nur ein anderes AGV detektiert, wartet das detektierende AGV. Bei einem Hindernis informiert die zentrale Einheit alle anderen AGVs darüber, dass dieser Streckenabschnitt blockiert ist. Das detektierende AGV ist allerdings nicht in der Lage den eigenen Pfad selbstständig zu korrigieren. An dieser Stelle muss manuell eingegriffen werden. 

\textcite{rfid} zeigen in ihrer Untersuchung exemplarisch, dass mit der RFID-Technologie die selbstorganisierte Entscheidungsfähigkeit der AGVs erhöht und die Abhängigkeit von einer zentralen Einheit reduziert werden kann.