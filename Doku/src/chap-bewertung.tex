\chapter{Bewertung}


\section{Zielerfüllung}

Die Zielstellung war die Veranschaulichung und Modularisierung eines Agenten-Systems in einer autonomen Fabrik mit Hilfe der Arduino- Plattform. Für diese Aufgabe wurde eine Liste erstellt, die sich aus Muss-, Soll, und Kann-Kriterien zusammensetzt. Damit das Projekt als erfüllt bewertet werden kann, sollten mindestens die Muss-Kriterien erfüllt sein.

Es sind alle Muss- und Soll-Kriterien erfüllt. Es wurden keine Kann-Kriterien umgesetzt. Damit ist das Projekt erfolgreich abgeschlossen und wurde etwas verfeinert.

hier noch ein paar beispiele.

\section{Probleme und Abweichungen}

Es gibt keine Abweichung zum Konzept. In der Umsetzung wurde geplant, dass der Kran nicht mit dem Server kommuniziert. Jedoch wird nun ein Bluetooth-Modul für den Kran verwendet, welcher mit dem Server kommuniziert. Die Kommunikation ist so nicht nur einheitlicher, sondern auch einfacher in das Agentensystem zu implementieren.
\todo[author=anja,inline]{das wurde im konzept nicht vorgesehen. im konzept haben wir sogar vergessen, dass der rodoter ja irgendwie signaliste bekommen muss, dass er wieder losfahren soll. du kannst es an sich als problem drin lassen. würde nur nicht sagen, dass es in der umsetzung geplant wurde, sondern eher, dass es sich jetzt so als günstiger erwiesen hat. das prob mit dem HC-Modul, was du erst reinschreiben wolltest, könntest du erwähnen (steht auch in der umsetzung). für nochn paar ideen zu problemen/abweichungen hilfts vlt konzept und umsetzung durchzulesen und die besprechungs/aufgabenaufzeichnungen von christian und vlt noch unsere präsentationen. da steht jedenfalls einiges drin, was wir mal an ideen hatten}

\section{Kritische Anmerkungen zum Projekt}\todo[author=anja]{würd ich persönlich mit der secion drüber zusammenführen}

Zur Veranschaulichung und Modularisierung eines Agenten-Systems ist die Arduino-Plattform gut geeignet. Die Komponenten können durch Zusammenstecken einfach miteinander verbunden werden. Auch die Anleitungen im Buch sind gut erklärt. Die Arduino-Software ist einfach zu installieren und wenig kompliziert.

Ein Nachteil beim Arbeiten mit der Arduino-Plattform ist, dass sich die zusammengesteckten Komponenten wieder leicht trennen. Dadurch müssen die Roboter oft überprüft werden, was einiges an Zeit kostet. Leider sind auch die Löcher auf den Plattformen für den Roboter falsch gestanzt. Damit der Kran und die Bohrer einen Roboter erkennen können, wurde ein Photowiderstand eingesetzt. Der Photowiderstand muss jedoch je nach Umgebungslicht angepasst werden. Das erweist sich als sehr problematisch, wenn die Agenten in sehr dunkler Umgebung oder in einem Umfeld mit sich ständig ändernden Lichtverhältnissen arbeiten müssen. \todo[author=anja,inline]{Dazu hatten wir die kalibrierung und die nachregelung mit potentiometer (klingt aktuell so, als obs die nicht gibt, probleme bestehen aber trotz dieser maßnahmen, weil regelung doch nicht so leicht geht). zur verbesserung wurde eine hysterese (unterschiedliche schwellwerte fürs erkennen rodoter da und erkennen rodoter nicht da, um schwanken zwischen rodoter nicht da - rodoter da im schwellwertbereich zu verhindern) vorgeschlagen, aber aus zeitgründen nicht umgesetzt} 

\section{Erweiterung des Projektes}

\todo[author=anja]{hier bietet sich ein kurzer satz dazu an, was unser projekt erreicht hat/zeigt}Als eine mögliche Erweiterung des Projektes eignet sich die Kommunikation zwischen den einzelnen Agenten. Statt einen gemeinsamen Server zur Kommunikation zu nutzen, sollen die Agenten untereinander kommunizieren können. Die Benutzung des Photowiderstandes zur Erkennung eines Roboters erwies sich in schwierigen Lichtverhältnissen als ungünstige Lösung. Ein Ziel kann sein, eine Alternative für den Photowiderstand zu finden.\todo[author=anja]{alternative lösungen z.b. lichtschranken oder kippschalter/bumper in der strecke, rifd wär vlt auch möglich (siehe theorieteil)}
\todo[author=anja, inline]{weitere erweiterungsmöglichkeiten: rfid-technologie(siehe theorieteil), produkt transportieren (z.b. magnetisch), leds für streckenunterbrechung statt breakpoints, kartendarstellung mit ortung in app, staumeldung(rodoter direkt nach borher liegen geblieben, keine rodoter fahren mehr zu diesem bohrer) }   