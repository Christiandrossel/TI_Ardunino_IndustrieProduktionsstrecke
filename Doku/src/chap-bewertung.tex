\chapter{Bewertung}

In diesem Kapitel wird das Projekt bewertet. Dazu sollte das Ergebnis mit der ursprünglichen Aufgabe/Zielstellung und auch dem Konzept abgeglichen werden. 

\section{Zielerfüllung}

Die Zielstellung ist die Veranschaulichung und Modularisierung eines Agenten-Systems in einer autonomen Fabrik mit Hilfe der Arduino- Plattform. Für diese Aufgabe wurde eine Liste erstellt, die sich aus Muss-, Soll, und Kann-Kriterien zusammensetzt. Damit das Projekt als erfüllt bewertet werden kann, sollten mindestens die Muss-Kriterien erfüllt sein.

Es sind alle Muss- und Soll-Kriterien erfüllt. Es wurden keine Kann-Kriterien umgesetzt. Damit ist das Projekt erfolgreich abgeschlossen und wurde etwas verfeinert.

\section{Abweichungen zum Konzept}

Abweichungen sollten diskutiert werden. 
Passt das mit den, was wir vorher uns vorgestellt haben? 

\section{Probleme in der Umsetzung}

Probleme während der Umsetzung können aufgegriffen und Bezüge zur Theorie hergestellt werden.
Gab es Probleme, wie Bluethooth module HC-05?
Die Umsetzung der Kann-Kriterien ist mit den gegebenen Mitteln sehr schwierig und fordert sehr viel Zeit.

\section{kritische Anmerkungen zum Projekt}

\subsubsection{positive Kritik}

Zur Veranschaulichung und Modularisierung eines Agenten-Systems ist die Arduino-Plattform gut geeignet. Die Komponenten können durch zusammenstecken einfach mit einander verbunden werden. Auch die Anleitungen im Buch sind gut erklärt. Die Arduino-Software ist einfach zu installieren und wenig kompliziert. 

\subsubsection{negative Kritik}

Ein Nachteil beim Arbeiten mit der Arduino-Plattform ist, dass sich die zusammengesteckten Komponenten wieder leicht trennen. Dadurch müssen die Roboter oft überprüft werden, was einiges an Zeit kostet. Leider sind auch die Löcher auf den Plattformen für den Roboter falsch gestanzt. 

\section{Erweiterung des Projektes}

Im letzten Abschnitt kann eine kurze Darstellung weiterführender Ideen und Möglichkeiten, die im Projekt nicht umgesetzt wurden, erfolgen. Wie könnte man das Projekt verbessern oder erweitern (im Hinblick auf theoretischen Rahmen - grober Bezug zur Zielstellung sollte vorhanden sein)? 

Die Vorschläge ergeben sich möglichst konsequent aus der Bewertung. 
Dieser Abschnitt kann auch als ein eigenes Kapitel Ausblick verfasst werden.