\chapter{Bewertung}

In diesem Kapitel wird das Projekt bewertet. Dazu sollte das Ergebnis mit der ursprünglichen Aufgabe/Zielstellung und auch dem Konzept abgeglichen werden. Abweichungen sollten diskutiert werden. Probleme während der Umsetzung können aufgegriffen und Bezüge zur Theorie hergestellt werden. Positive wie kritische Anmerkungen gehören gleichermaßen in dieses Kapitel.

Es sollte berücksichtigt werden, dass ein Leser möglicherweise nicht alle vorangegangenen Kapitel gelesen hat.

Im letzten Abschnitt kann eine kurze Darstellung weiterführender Ideen und Möglichkeiten, die im Projekt nicht umgesetzt wurden, erfolgen. Wie könnte man das Projekt verbessern oder erweitern (im Hinblick auf theoretischen Rahmen - grober Bezug zur Zielstellung sollte vorhanden sein)? Die Vorschläge ergeben sich möglichst konsequent aus der Bewertung. Dieser Abschnitt kann auch als ein eigenes Kapitel Ausblick verfasst werden.