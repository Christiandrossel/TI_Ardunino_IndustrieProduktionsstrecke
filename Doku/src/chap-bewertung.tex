\chapter{Bewertung}


\section{Zielerfüllung}

Die Zielstellung war die Veranschaulichung und Modularisierung eines Agenten-Systems in einer autonomen Fabrik mit Hilfe der Arduino- Plattform. Für diese Aufgabe wurde eine Liste erstellt, die sich aus Muss-, Soll, und Kann-Kriterien zusammensetzt. Damit das Projekt als erfüllt bewertet werden kann, sollten mindestens die Muss-Kriterien erfüllt sein.

Es sind alle Muss- und Soll-Kriterien erfüllt. Es wurden keine Kann-Kriterien umgesetzt. Damit ist das Projekt erfolgreich abgeschlossen und wurde etwas verfeinert.

hier noch ein paar beispiele.

\section{Probleme und Abweichungen}

Es gibt keine Abweichung zum Konzept. In der Umsetzung wurde geplant, dass der Kran nicht mit dem Server kommuniziert. jedoch wird nun ein Bluetooth-Modul für den Kran verwendet, welcher mit dem Server kommuniziert. Die Kommunikation ist so nicht nur einheitlicher, sondern auch einfacher in das Agentensystem zu implementieren.

\section{kritische Anmerkungen zum Projekt}

Zur Veranschaulichung und Modularisierung eines Agenten-Systems ist die Arduino-Plattform gut geeignet. Die Komponenten können durch zusammenstecken einfach mit einander verbunden werden. Auch die Anleitungen im Buch sind gut erklärt. Die Arduino-Software ist einfach zu installieren und wenig kompliziert. 

Ein Nachteil beim Arbeiten mit der Arduino-Plattform ist, dass sich die zusammengesteckten Komponenten wieder leicht trennen. Dadurch müssen die Roboter oft überprüft werden, was einiges an Zeit kostet. Leider sind auch die Löcher auf den Plattformen für den Roboter falsch gestanzt. Damit der Kran und die Bohrer einen Roboter erkennen können, wurde einen Photowiderstand eingesetzt. Der Photowiderstand muss jedoch je nach Umgebungslicht angepasst werden. Das erweist sich als sehr problematisch, wenn die Agenten in sehr dunkler Umgebung oder in einem Umfeld mit sich ständig ändernden Lichtverhältnissen arbeiten müssen. 

\section{Erweiterung des Projektes}

Als eine mögliche Erweiterung des Projektes eignet sich die Kommunikation zwischen den einzelnen Agenten. Statt einen gemeinsamen Server zur Kommunikation zu nutzen, sollen die Agenten untereinander kommunizieren können. Die Benutzung des Photowiderstandes zur Erkennung eines Roboters erwies sich in schwierigen Lichtverhältnissen als ungünstige Lösung. Ein Ziel kann sein, eine Alternative für den Photowiderstand zu finden.   