\pseudochapter{Kurzfassung}


Dieses Dokument beschreibt die Umsetzung einer modularen Produktionsstrecke nach den Vorbild Industrie 4.0.
Gegenstand der Arbeit ist ein Model, in dem Roboter Werkstücke
von einen Kran zu einer freien Bohrstationen bringen und von da aus wieder zum Kran zurück. Dieses Modell soll dazu dienen Abläufe zu erklären und Schwierigkeiten zu erkennen und diese zu lösen.
Die Kommunikation zwischen den Robotern und Stationen erfolgt über Bluetooth. Dabei basieren die einzelnen Geräte auf der Arduino Plattform. Diese sind über einen Android Server zentral verbunden.
Als Ergebnis dieser Arbeit entstand ein Modell einer funktionierende Industrie 4.0 Produktionsstrecke. Es werden zudem mögliche Verbesserungen oder Erweiterungen Diskutiert.
