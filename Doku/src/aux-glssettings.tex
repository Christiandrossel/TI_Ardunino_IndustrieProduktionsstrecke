% default style
\setglossarystyle{long}

% symbols
\newglossary[slg]{symbols}{syi}{syg}{Symbols}
%\renewcommand*{\glspostdescription}{}

% make glossaries
\makeglossaries{}

% symbol style
\newglossarystyle{symbol}{%
    % 'longtable' with 3 columns
    \renewenvironment{theglossary}%
    {\begin{longtable}{llp{\glsdescwidth}}}%
    {\end{longtable}}%
    % Table head
    \renewcommand*{\glossaryheader}{%
        \sffamily\bfseries Symbol & \sffamily\bfseries Einheit & %
        \sffamily\bfseries Beschreibung \\\endhead}%
    % no space between groups
    \renewcommand*{\glsgroupheading}[1]{}%
    %
    \renewcommand*{\glossaryentryfield}[5]{%
        % \glsentryitem{##1}% Entry number if required
        \glstarget{##1}{##2}% Name
        & ##4 %Symbol
        & ##3% Description
        \\% end of row
    }%
    % The command \glsgroupskip specifies what to do between glossary groups.
    % Glossary styles must redefine this command. (Note that \glsgroupskip
    % only occurs between groups, not at the start or end of the glossary.)
    \renewcommand*{\glsgroupskip}{\relax}
}
