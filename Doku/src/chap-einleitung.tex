\chapter{Einleitung}

%Einführung in das Thema und Darstellung des Projektziels à la \enquote{Sie haben schon lange Einzug in den Haushalt in allen sozialen Schichten gehalten und sind fester Bestandteil einer jeden Feierlichkeit: Kuchen. Die Variationen des Kuchenbackens sind vielfältig. Ein aktueller Trend\dots{} bla bli blubb\dots{} Ziel dieses Projekts ist es daher, ein Modell für smartes Kuchenbacken zu entwickeln\dots{}}


In der Industrie existieren stets Bestrebungen Innovation einzuführen, die Wirtschaftlichkeit zu verbessern und den Menschen als Fehlerquelle auszuschließen. Dies wird insbesondere ausgedrückt durch die dafür geschaffenen Kunstwörter der Industriellen Revolution. Den Beginn markiert der erste Einsatz von Massenproduktion (Industrie 1.0), ein weiterer Schritt ist Elektrizität (Industrie 2.0) und der Einsatz von Computern (Industrie 3.0), bis zur heutigen Digitalisierung und Weiterentwicklung (Industrie 4.0)\autocite{industrie}. Während die ersten drei Revolutionen im Nachgang als solche bezeichnet wurden, wird die aktuelle Industrie 4.0 bereits vor ihrer Vollendung als Revolution bezeichnet. Vor allem der Wirtschaftsstandort Deutschland, der den Begriff geprägt hat, möchte die eigene Wirtschaft in einer Spitzenposition sehen \autocite{arbeitsbericht}. Aber auch andere Länder sind dabei einen Wandel der Wirtschaft herbeizuführen. Weltweit lässt sich dieses Phänomen durch die Entstehung von \enquote{Smart Factories} beobachten. Doch auch dort ist die \enquote{Intelligente Fabrik} der Industrie 4.0 erst am Beginn der Entwicklung. Die Wirtschaft ist also noch in einem Findungsprozess, der mithilfe von Digitalisierung, vor allem von Arbeitsschritten, sowie dem Einsatz von Vernetzung, wie mit dem Internet der Dinge geschehen , und auch durch Bildung von Initiativen und Plattformen beschleunigt werden soll.

In diesen Prozess steigt die Projektarbeit durch Entwicklung einer physischen Simulation einer Industrie 4.0 Produktionskette ein. Dabei soll das Projekt stärker den Kommunikationsaspekt der Industrie 4.0 aufgreifen, anstatt eine Produktionsstrecke mit vollständig gefertigtem Produkt darzustellen. Aufbauen soll das Projekt auf einem aktuellen Forschungsthema der Selbstorganisation von Prof. T. Munkelt und M.Sc. M. Krockert der Hochschule für Technik und Wirtschaft Dresden(HTW) \autocite{agents}.

Die Basis aller digitalisierten Teilnehmer der Produktionskette bildet der Mikrocontroller Arduino. Mit diesem Mikrocontroller können elektrische Bauteile, wie LEDs oder Motoren angesteuert werden. Mithilfe von drahtlosen Datenübertragungsverfahren wie Bluetooth oder WiFi können die verwendeten Mikrocontroller untereinander kommunizieren, was eine Voraussetzung für Industrie 4.0 darstellt. Der Vorteil des Arduinos als Mikrocontroller besteht in seiner geringen Größe mittels der er portabel ist. Somit kann die entwickelte Lösung unter Verwendung von Arduinos \todo[author=timon, inline]{doppelt evtl weglassen} zur Vorstellung des Forschungsthemas, beispielsweise bei Messen und Konferenzen, genutzt werden. 