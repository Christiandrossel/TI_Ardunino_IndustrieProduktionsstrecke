\chapter{Einleitung}

%Einführung in das Thema und Darstellung des Projektziels à la \enquote{Sie haben schon lange Einzug in den Haushalt in allen sozialen Schichten gehalten und sind fester Bestandteil einer jeden Feierlichkeit: Kuchen. Die Variationen des Kuchenbackens sind vielfältig. Ein aktueller Trend\dots{} bla bli blubb\dots{} Ziel dieses Projekts ist es daher, ein Modell für smartes Kuchenbacken zu entwickeln\dots{}}

In jedem Jahr gibt es neue Ideen die Industrie zu revolutionieren, wirtschaftlich zu verbessern, sowie die Fehlerquelle Mensch zu ersetzen. Dies wird insbesondere ausgedrückt durch die dafür geschaffenen Kunstwörter der Industriellen Revolution. Angefangen hat es mit dem ersten Einsatz von Massenproduktion (Industrie 1.0), Elektrizität (Industrie 2.0) und Computer (Industrie 3.0), geht über Digitalisierung und Weiterentwicklung (Industrie 4.0)\footnote{https://www.dfki.de/wwdata/Zukunft\_der\_Industrie\_IHK\_Darmstadt\_22\_01\_2015/Industrie\_4\_0\\\_Das\_Internet\_der\_Dinge\_kommt\_in\_die\_Fabriken\_Copyright.pdf}, bis zu Künstlicher Intelligenz und Biotechnologie (Industrie 5.0). Während die ersten drei Revolutionen im Nachgang als solche betitelt wurden, wird die aktuelle Industrie 4.0 schon im Verlauf als Revolution bezeichnet. Vor allem der Wirtschaftsstandort Deutschland, der den Begriff geprägt hat, möchte die eigene Wirtschaft in einer Spitzenposition sehen.\footnote{https://www.acatech.de/publikation/umsetzungsempfehlungen-fuer-das-zukunftsprojekt-industrie-4-0-abschlussbericht-des-arbeitskreises-industrie-4-0/} Aber auch andere Länder sind dabei einen Wandel der Wirtschaft herbeizuführen. Weltweit lässt sich dieses Phänomen durch die Entstehung von \enquote{Smart Factories} beobachten. Doch auch dort ist die \enquote{Intelligente Fabrik} der Industrie 4.0 erst am Beginn der Entwicklung. Die Wirtschaft ist also noch in einem Findungsprozess, welcher mithilfe von Digitalisierung, vor allem von Arbeitsschritten sowie dem Einsatz von Vernetzung, wie dem Internet der Dinge geschehen, als auch durch Bildung von Initiativen und Plattformen beschleunigt werden soll.\\
In diesen Prozess soll die Projektarbeit durch Entwicklung einer physischen Simulation einer Industrie 4.0 Produktionskette einsteigen. Dabei soll das Projekt stärker den Kommunikationsaspekt der Industrie 4.0 aufgreifen anstatt eine Produktionsstrecke mit vollständig gefertigtem Produkt darzustellen. Aufbauen soll das Projekt auf einem aktuellen Forschungsthema der Selbstorganisation von Prof. T. Munkelt und M.Sc. M. Krockert der Hochschule für Technik und Wirtschaft Dresden(HTW) \enquote{AGENT-BASED SELF-ORGANIZATION VERSUS CENTRAL PRODUCTION PLANNING}.\footnote{https://www.informs-sim.org/wsc18papers/includes/files/287.pdf}\\
Die Basis aller digitalisierten Teilnehmer der Produktionskette bildet der Mikrocontroller Arduino. Mit diesem Mikrocontroller können elektrische Bauteile, wie LEDs oder Motoren angesteuert werden. Mithilfe von Bluetooth oder WiFi können die verwendeten Mikrocontroller untereinander kommunizieren, was eine Voraussetzung für Industrie 4.0 darstellt. Der Vorteil des Arduinos als Mikrocontroller besteht in seiner geringen Größe mittels der er portabel ist und somit auf Messen mit dem Forschungsthema ausgestellt werden kann.  