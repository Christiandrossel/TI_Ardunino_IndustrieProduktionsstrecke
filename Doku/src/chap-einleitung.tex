\chapter{Einleitung}

%Einführung in das Thema und Darstellung des Projektziels à la \enquote{Sie haben schon lange Einzug in den Haushalt in allen sozialen Schichten gehalten und sind fester Bestandteil einer jeden Feierlichkeit: Kuchen. Die Variationen des Kuchenbackens sind vielfältig. Ein aktueller Trend\dots{} bla bli blubb\dots{} Ziel dieses Projekts ist es daher, ein Modell für smartes Kuchenbacken zu entwickeln\dots{}}


In der Industrie existieren stets Bestrebungen Innovation einzuführen, die Wirtschaftlichkeit zu verbessern und den Menschen als Fehlerquelle auszuschließen. Dies wird insbesondere ausgedrückt durch die dafür geschaffenen Kunstwörter der Industriellen Revolution. Den Beginn markiert der erste Einsatz von Massenproduktion (Industrie 1.0), gefolgt von\todo[author=anja]{vorher entstand gefühlt ein stilbruch} Elektrizität (Industrie 2.0) und dem\todo[author=anja]{grammatikanpassung} Einsatz von Computern (Industrie 3.0), bis zur heutigen Digitalisierung und Weiterentwicklung (Industrie 4.0; \nptextcite{industrie})\todo[author=anja]{schicker in einer klammer}. Während die ersten drei Revolutionen im Nachgang als solche bezeichnet wurden, wird die aktuelle Industrie 4.0 bereits vor ihrer Vollendung als Revolution bezeichnet. Vor allem der Wirtschaftsstandort Deutschland, der den Begriff geprägt hat, möchte die eigene Wirtschaft in einer Spitzenposition sehen \autocite{arbeitsbericht}. Aber auch andere Länder sind dabei einen Wandel der Wirtschaft herbeizuführen. Weltweit lässt sich dieses Phänomen durch die Entstehung von \enquote{Smart Factories} beobachten. Doch auch dort steht\todo[author=anja]{schicker als ist} die \enquote{Intelligente Fabrik} der Industrie 4.0 erst am Beginn ihrer\todo[author=anja]{schicker als der} Entwicklung. Die Wirtschaft ist also noch in einem Findungsprozess, der mithilfe von Digitalisierung, vor allem von Arbeitsschritten, sowie dem Einsatz von Vernetzung, wie mit dem Internet der Dinge geschehen, und auch durch Bildung von Initiativen und Plattformen beschleunigt werden soll.

In diesen Prozess steigt die Projektarbeit durch Entwicklung einer physischen Simulation einer Industrie 4.0 Produktionskette ein. Dabei wird stärker der Kommunikationsaspekt der Industrie 4.0 aufgegriffen, als eine Produktionsstrecke mit vollständig gefertigtem Produkt dargestellt. Das Projekt baut auf dem\todo[author=anja]{dem statt einem, da konkretes thema} aktuellen Forschungsthema der Selbstorganisation von \todo[author=anja]{titel und initialen (ausnahme namensdoppelungen) schreibt man nicht mit rein} Munkelt und Krockert der Hochschule für Technik und Wirtschaft Dresden (\citeyear{agents})\todo[author=anja]{getrickst} auf.
\todo[author=anja,inline]{die solls, sollten laut rückmeldung hier ja raus (vlt damit es paperartiger klingt. ansonsten ist soll schon die richtige form für die formulierung einer anforderung). ich habs jetzt so geändert, dass die solls rausgefolgen sind.}
\todo[author=anja,inline]{ abkürzung kann weg, wenn nicht wiederverwendet. institution kann eig auch weggelassen werden, da sie im weiteren verlauf keine rolle spielt. dann textcite verwenden für zitation in textform}
Die Basis aller digitalisierten Teilnehmer der Produktionskette bildet der Mikrocontroller Arduino. Mit diesem Mikrocontroller können elektrische Bauteile, wie LEDs oder Motoren angesteuert werden. Mithilfe von drahtlosen Datenübertragungsverfahren wie Bluetooth oder WiFi können die verwendeten Mikrocontroller untereinander kommunizieren, was eine Voraussetzung für Industrie 4.0 darstellt. Der Vorteil des Arduinos als Mikrocontroller besteht in seiner geringen Größe mittels der er portabel ist. Somit kann die entwickelte Lösung unter Verwendung von Arduinos \todo[author=timon, inline]{doppelt evtl weglassen (anja: jo, lass weg)} zur Vorstellung des Forschungsthemas, beispielsweise bei Messen und Konferenzen, genutzt werden. 