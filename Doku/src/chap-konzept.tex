\chapter{Konzept}

Vorbereitungen zum Projekt wie die StarterKit-Projekte (welche bearbeitet, Probleme, Auswahl an Modifikationen) oder die Verwendung von Notepad++ als externen Editor. Vorstellung des Konzepts und der Überlegungen dazu. Achtung: nicht vermischen mit der konkreten Umsetzung.

\section{Vorbereitung}
Einleitender Text
\subsection{Arduino StarterKit-Projekte}
Kurze Erklärung worums geht

\subsubsection{Farbmischende Lampe}
Ich habe mich entschieden die RGB-LED für mein Projekt zu verwenden. Ich will, dass diese LED in verschiedenen Farbtönen aufleuchtet. Außerdem soll der Benutzer entscheiden können, welche Farbe gerade aufleuchten soll. Genauso soll er das Licht auch einfach wieder ausschalten können. Die Auswahl des Farbtones soll sehr groß sein. Ich möchte, das die LED einfach zu bedienen ist und nicht für jeden Farbton ein einzelner Schalter angelegt werden muss.

\subsubsection{Alarmanlage}
Text

\subsubsection{Keyboard}
Text

\subsubsection{Digitale Sanduhr}
Text

\subsubsection{Windrad}
Für das Projekt Windrad werden neben einem Gleichstrommotor, eine Diode, MOSFET Transistor, 9-Volt-Block Batterie und ein 100 $\mu$F Kondensator mit Potentiometer verwendet. In dem Einstiegsprojekt aus dem Buch war es bisher nur möglich den Motor, mit einem Taster ein- und auszuschalten. Mit diesem Umbau kann der Nutzer durch drehen am Potentiometer selbst entscheiden, mit welcher Geschwindigkeit der Motor und damit das Windrad drehen soll.

\subsubsection{Uhr}
Text